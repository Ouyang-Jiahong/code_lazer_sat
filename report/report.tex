\documentclass[openany,12pt,UTF8]{ctexart}
\usepackage[a4paper,margin=2.5cm]{geometry}
\include{package}
\title{程序说明}
\author{欧阳嘉鸿}
\date{\today}
\begin{document}
\maketitle
\newpage
\tableofcontents
\newpage
\section{数学模型}

\subsection{决策变量}
定义如下布尔型决策变量:
$$
    x_{r,s,a} =
    \begin{cases}
        1, & \text{若雷达 } r \text{ 在其第 } a \text{ 个可见弧段对目标 } s \text{ 进行观测} \\
        0, & \text{否则}
    \end{cases}
$$
其中:$r \in R$(雷达集合),$s \in S$(目标集合),$a \in A_{r,s}$(雷达 $r$ 对目标 $s$ 可观测的弧段索引集合)

\subsection{目标函数}
首先,我们定义以下两个关键概念:

\textbf{有效观测}:若雷达 $r$ 对目标 $s$ 的一次观测满足观测时长 $t_{r,s}$ 大于目标要求的最小观测时长 $T_s^{\min}$,则称这次观测为有效观测。

\textbf{有效观测目标}:若目标 $s$ 从 $N_r$ 个不同雷达处共获得 $N_s$ 次有效观测,并且 $N_r$ 不小于目标要求的最小测站数量 $M_s^{\min}$,且 $N_s$ 不小于目标要求的最小有效观测次数 $N_s^{\min}$,则称目标 $s$ 为有效观测目标。

目标函数 $J$ 表示所有有效观测目标数量的加权和:
$$
    \max J = \sum_{s \in S} w_s \cdot y_s
$$
其中:
\begin{itemize}
    \item $w_s$:目标 $s$ 的权重(优先级);
    \item $y_s$:二值变量,表示目标 $s$ 是否为有效观测目标。
\end{itemize}

\subsection{约束条件}

\subsubsection{雷达观测容量约束}
每部雷达在任一时刻最多只能同时观测 $C_r$ 个目标。考虑离散时间索引 $t \in T$,有:
$$
    \sum_{s \in S} \sum_{a \in A_{r,s}^t} x_{r,s,a} \leq C_r, \quad \forall r \in R,\ \forall t \in T
$$
其中:
\begin{itemize}
    \item $A_{r,s}^t$:时刻 $t$ 时,雷达 $r$ 可用于观测目标 $s$ 的弧段集合
    \item $T$:时间步长集合
\end{itemize}

以上模型描述了一个带有可见性窗口和有效时长判据的多雷达-多目标分配问题,旨在在资源受限下最大化目标的有效观测收益。

\end{document}